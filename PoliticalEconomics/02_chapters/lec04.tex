In this and the next lecture we will shift to another important question, namely what determines the level of redistribution in society, and why this differs so much across time and countries. In this section we discuss the Meltzer-Richard model. This is a classical economists model which assumes rational voters with their own utility in mind when they vote. Naturally this approach yields a conclusion along the lines of an income-based theory for preferences to redistribution in which the relative income determines ones preferences for redistribution. Combining this with the median voter theorem, the difference between mean and median income becomes important for determining the level of redistribution.

\paragraph{Model setup}
The model setup follows the one in section \ref{seq: lecture1}, voters occupy a continuum and each has utility $w^i = c^i + V(x^i)$ subject to the budget constraint $c^i = (1-\tau)I^i + f$ and the time constraint $1+e^i=I^i + x^i$\footnote{Here we add $e^i$, in section \ref{seq: lecture1} we subtracted $\alpha^i$ so the sign is flipped.}. $e^i\sim F(\cdot)$ with mean $e$ and median $e^m$. Just like in section \ref{seq: lecture1} we can solve the voters utility maximization by inserting the constraints in the utility and maximizing w.r.t $I^i$.
\\ \\ 
Once again we can derive a FOC of $I^i = 1 + e^i - V_x^{-1}(1-\tau)$ and we can define the average labor supply of the average individual as 
\begin{equation}
    L(\tau) = 1 + e - V_x^{-1}(1-\tau)
\end{equation}
implying the FOC can be rewritten $I^i = L(\tau)+(e^i - e)$. The government follows a budget constraint $f=\tau I=\tau L(\tau)$ and the indirect utility is then given by 
\begin{equation}
    W^i(\tau) = L(\tau) + (1-\tau)(e^i - e) + V(1-L(\tau) + e)
\end{equation}

Taking the derivative of this and setting equal to 0 we can then solve for $\tau$ to derive the individuals bliss point $\tau^i$, first take the derivative 
\begin{equation}
    \frac{\partial W^i}{\partial \tau} = L_{\tau}(\tau) - (e^i - e) - V'(1-L(\tau) + e)L_{\tau}(\tau)
\end{equation}
From the FOC we have that in equilibrium $V'(1-L(\tau) + e) = 1- \tau$. Using this and rearranging gives us the bliss point 
\begin{equation}
    \tau^i = \frac{e-e^i}{-L_{\tau}(\tau)}
\end{equation}
Since $L_{\tau}(\tau)<0$ this expression is decreasing in $e^i$. For all individuals with $e^i>e$ therefore prefer negative taxes while those with $e^i<e$ prefers positive taxes. 

\paragraph{Downsian voting} As we saw in section \ref{seq: lecture1} this model satisfies the single crossing property, so pure majority voting will ensure that the condorcet winner is chosen, and this will be the preferred tax rate of the median voter $\tau^m=\frac{e-e^m}{-L_{\tau}(\tau)}$. This expression shows the main conclusion; more inequality increases the difference between $e$ and $e^m$ yielding a higher tax rate. Notice this is not the same as saying income inequality causes more redistribution, if for example the income distribution skews because the middle class gains income, this raises $e$ along with $e^m$. Oppositely if the rich get extremely rich, this changes $e^m$ without affecting $e$ a lot. Also note that the denominator $|L_{\tau}(\tau)|$ is the change in labor supply as a consequence of taxation, i.e. the deadweight loss of taxation. A higher cost of taxation reduces the desire for redistribution because individuals forego personal income in a tradeoff with the size of the public transfer. 
\\ \\ 
Also take note that like before $e^m$ is the median productivity of the average voter while $e$ is the average income of the average \emph{tax payer}. As a consequence extending voting rights to poorer citizens will decrease $e^m$ but not affect $e$, resulting in more redistribution.

\subsection{Empirical results}
A first consideration is that our model assumes that higher income leads to lower desired taxes. However when regressing preferences for distribution on log income we observe a significant but not very predictive relationship, showing that many other variables also matter. With this in mind let us consider some empirical evidence investigating the conclusions from our model. Furthermore the model really compares the amount of redistribution to the would-be income distribution if a nation had no government at all. This is of course very difficult to observe meaning researchers have to come up with alternative measures of the "baseline inequality". 
\\ \\ 
In general the evidence for a relation between the skewness of a nations income distribution and the amount of government redistribution is mixed. Some find a positive correlation while others dont. \cite{aidt_democracy_2006} investigates the consequence of "franchice-extentions" (e.g. broader voting rights for the poor/women) in 12 european countries and find that broadening the voter base to include poorer voters does induce an increase in government spending. This finding is consistent with our model. However their result suggest that the added spending is mainly directed towards infrastructure and internal security, which is not obviously increase redistribution, as our model would suggest. 

\cite{alesina_why_2001} investigates perhaps the largest puzzle in the voting-spending question, namely how comes the european countries have build large welfare states, while the US remains a relatively slim state with limited spending, especially on redistributive efforts. The authors suggests explanations can be put in one of three bins: "economic", "political" and "behavioral" explanations. 
\\ \\ 
Clearly the simple model is not able to explain the differences between Europe and the US, as the income distribution is more skewed in the US, which should suggest they had a higher level of redistribution. One extension considered by \citeauthor{alesina_why_2001} is that income dynamics means $y^i$ is not static, but can change over the life cycle. If the median voter expects $y^i$ to increase in the future, this would dampen the demand for redistribution, as it 1) might reduce the possibility to climb the income ladder and 2) will be costly once the median voter earns a higher income. However there is little empirical evidence to support the idea that income is more upwards mobile in the US than in Europe. Americans do however believe more in the idea of upwards mobility than europeans do. 
\\ \\ 
Another way to modify the model is to introduce altruistic agents. By assuming europeans care more for the well being of the poor than europeans one gets a straight forward explanation for the observed differences in redistribution. \citeauthor{alesina_why_2001} give two explanations for why this might be the case. First of the USA is more racially fractionalized than Europe, and it is well established within psychology that people identify with groups, and that race is an easily visible marker for "group" membership. Especially considering that minorities are over-represented among the poor in the USA, race might have become a group marker, justifying lower redistribution (although this would probably require that the politicians were primarily elected by whites, as otherwise minority votes should affect the policy in equal amount). 

Another behavioral explanation is that americans and europeans differ in their beliefs about welfare recipients, so that americans are more likely to believe that social welfare recipients are lazy. There is some evidence that this is the case from survey questions.

