In the previous section we studies the simple Meltzer-Richard model predicting that a more skewed income distribution (where the mean was measured among the voters) would result in more government redistribution. This model has a major issue, namely that it's conclusions are exactly opposite of what we observe when comparing the US and Europe. The paper by \citeauthor{alesina_why_2001} debated solutions to this "paradox". They propose that income mobility or perceptions about it can shape voters attitudes towards redistribution, that racial divides and "group thinking" might be to blame for low redistribution in the US, or that beliefs about the causes of poverty were central in shaping voters preferences for redistribution. 
\\ \\
In this section we will focus on individual level preferences for redistribution. In particular we will attempt to answer three questions related to this, each studies by a separate paper 
\begin{itemize}
    \item Are people even aware of the degree of inequality and redistribution in society? \citep{gimpelson_misperceiving_2018}
    \item Are peoples preferences towards redistribution affected by their knowledge about the current level of inequality and redistribution? \citep{kuziemko_how_2015} 
    \item Do beliefs about social mobility play into these questions? \citep{alesina_why_2001} 
\end{itemize}

\subsection{Misperceiving inequality, \citep{gimpelson_misperceiving_2018}}
In the Meltzer-Richard model we assume that individuals are aware of the distribution of incomes in society, and that they know exactly their own position in the distribution. This entails a conclusion that individuals care about their relative position in the income distribution $e-e^i$ when determining what level of redistribution they would prefer. The paper by \citeauthor{gimpelson_misperceiving_2018} asks to what degree individuals are even aware of their own place in the income distribution. They find that ordinary people generally does not know how they fit in the distribution, suggesting that theories that relies on some relation between income inequality and politics fails at a very basic level. When using individuals perceived income inequality instead of actual inequality, there is a clear relation between income (perceived) and the demand for redistribution. 
\\ \\ 
\citeauthor{gimpelson_misperceiving_2018} use survey dataset from several countries to elicit individuals knowledge of the income distribution in their country. They find that in almost all countries resident guesses the average income quite wrong. In most countries the average guessed mean income was even on the wrong side of the median income. Furthermore peoples guesses about the distributional shape are very varying, with the most common answer getting less than 50\% of total answers in 29 of 40 countries. 
\\ \\
Plotting actual and perceived GINI index against each other shows little or no correlation. The authors also asks respondents if they believe income inequality has increased or decreased over the past 5 years, and find that regardless of the actual development most people guess that inequality has increased. (This result is somewhat questionable, first of people rarely think about the economy in 5-year periods, and the distinction between income and wealth becomes blurry when asking ordinary people)

\paragraph{Life in transition survey} The authors main point, that knowledge about ones position in the income distribution is very limited is however quite strong, as it seems people at both the top and bottom of the income distribution tend to estimate their position closer to the center than they are. In LiTS respondents are asked if they believe government should redistribute income between people, a measure which the authors regress using their "perceived position" measure while controlling for actual country GINI. They find that perceived GINI is significant in explaining respondents attitude towards redistribution, implying that a perceived higher inequality is associated with an increased interest in government redistribution.

\subsection{How elastic are preferences for redistribution? \citep{kuziemko_how_2015}}
The next paper by \citeauthor{kuziemko_how_2015} digs deeper into the \textit{causal} link between perceived inequality and preferences for redistribution.