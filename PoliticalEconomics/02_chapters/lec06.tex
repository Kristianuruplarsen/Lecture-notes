This lecture goes into the paper by \cite{alesina_intergenerational_2018}. This paper studies the effect of perceived income mobility on attitudes towards redistribution. We have already seen evidence that americans are more optimistic about income inequality than europeans, but whether this optimism causally influences attitudes politics is unclear. The authors collect survey data from France, Italy, Sweden, UK and the US to get a cross country picture of the relation between perceived income mobility and demand for redistribution. Furthermore the authors add an element of randomization to the questionnaires to manipulate the respondents beliefs about income mobility. They also ask for respondent political beliefs allowing them to estimate any heterogeneity across the political spectrum. 
\\ \\
Unlike \cite{alesina_why_2001} which focuses on the intragenerational mobility, \cite{alesina_intergenerational_2018} focus on intergenerational mobility. The authors use data on parents incomes matched to data on childrens income when adult to calculate transition probabilities towards each quintile for children with parents in the bottom quintile. (I.e. what is the probability of being born in the bottom five and ending up in the fourth quintile). 
\\ \\
In the survey questions respondents are then asked to guess these probabilities. Comparing the actual numbers to the average questionaire answers. This shows that US citizens underestimate the probability of remaining in the bottom quartile while overestimating the probability of going from Q1 to Q5. In Europe respondents guess exactly opposite of this.

Correlating peoples guess of Q1 to Q5 income mobility with peoples preference for various redistributive policies show that believing more people make the Q1 to Q5 transition imply lower support for redistribution. To understand if this relation is causal the authors use their RCT setup in which some of the respondents are shown videos about income mobility designed to make them more pessimistic towards the level of mobility. They find the treatment significantly alter policy preferences for left-wing voters, while right wing voters change their perception of the issue but not their policy views. One way to interpret these results is that right wing voters have lower trust in the government, which is reinforced by the treatment videos, making them distrust the effectiveness of policy. 