In this lecture we will see a number of articles debating to what extend real politics operate after affect or elect mechanics. Essentially this is a question of whether it matters which party wins the election or not. To test this empirically we need to study the evolution of policies over election cycles. A major identification issue in this question is that the level of affecting and electing might vary across parties and time, including over the business cycle. One way to get around this is to use a RD design to compare close elections.

\subsection{Do political parties matter? Evidence from US cities}
The first paper estimates an RD model on election data from local elections in the US \citep{ferreira_political_2007}. They show the mayors political affiliation does not affect the size of government, the allocation of spending or on local crime rates. However they do find a significant incumbents advantage. For data they use information on mayoral election and policy outcomes from 1950 to 2005. Their sample covers all US cities with more than 50.000 citizens. 

This first paper seems to suggest there is no real elect element in mayor election, but this does not rule out the affect mechanism. Furthermore one might argue mayorial elections are low-attention low-stakes elections compared to national and statewide elections. 

\subsection{Do voters affect or elect policies? Evidence from the US house}
The paper by \cite{lee_voters_2004} does almost the same as the paper by \citeauthor{ferreira_political_2007}, they set up a RD design but instead of mayoral elections they use data from the US House of Representatives. For outcomes they use the individual politicians ADA scores (a kind of "policy score"). In their dataset the RD shows a large discontinuity in voting behavior between democrats and nondemocrat politicians. This thus implies that the elect channel is very important in House elections. Furthermore using some clever econometrics they show that in fact the affect mechanism is of almost no importance, suggesting full policy divergence. 
\\ \\
The way they do this is by noting that if affect is a mechanism at play we would expect an increase in democrats electoral strength to cause both parties to propose more left-wing policies. One way to think of this is that if many incumbents are democrats, they have an advantage and republicans have to make up this difference by moving their proposals for $g$ towards $g^m$. So a democrat incumbent would result in more leftist policies after the election both by a) democrats being more likely to be reelected (the elect channel) and b) republicans proposing more leftist policy as well (the affect channel). 

The first part can be estimated using a RD to estimate the increase in reelection probability and any change in period $t$ voting behavior caused by incumbency. The authors can also estimate the total effect of by estimating the causal effect of incumbency on period $t+1$ voting behavior regardless of who gets elected. They can then residually calculate the affect component. 


\subsection{Do parties matter for economic outcomes? A regression discontinuity approach}
The final paper for this lecture is by \cite{pettersson-lidbom_parties_2008}. Here the authors do pretty much the same as in \cite{ferreira_political_2007}, but using data for Swedish local government elections. The author finds a significant party effect on expenditures over total income, unemployment rates and the number of public servants. All effects go in the expected direction so for example left wing mayors have on average more public servants and higher expenditures than right wing mayors.


\subsection{Summary}
Of the three papers we've seen, two study the effect of partisanship on actual policy outcomes, while \cite{lee_voters_2004} instead considers the effect of partisanship on voting behavior in parliament. Both of the two papers studying the effect on actual policy outcomes use local elections, \cite{ferreira_political_2007} in the US and \cite{pettersson-lidbom_parties_2008} in Sweden. In the US there seems to be no effect of partisanship (i.e. full policy convergence) while the Swedish data reveals large differences (little or no convergence). So at least in the US it is quite clear that national politics are driven primarily by the elect mechanism, while it is less certain what (if anything) matters for local elections. It would appear that the setting of elections matter a lot, but how or why is not clear.
