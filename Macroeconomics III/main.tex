\documentclass[12pt, a4paper]{article}
\usepackage{marginnote}
\usepackage[top=2.5cm, bottom=2.5cm, left=3.3cm, right=3.3cm, marginparwidth = 2.8cm]{geometry}
\usepackage[utf8]{inputenc}
\usepackage[english]{babel} 
\usepackage[T1]{fontenc} 
\usepackage{amsmath,amssymb} 
\usepackage[table]{xcolor}
\usepackage{verbatim}
\usepackage{graphicx}
\usepackage{fancyhdr}
\usepackage{wrapfig}
\usepackage{gauss}
\usepackage{xfrac}
\usepackage[page]{totalcount}
\usepackage{caption}
\usepackage{booktabs}
\usepackage{setspace}
\usepackage{enumitem}
\usepackage{placeins}
\usepackage{hyperref}
\usepackage{mathtools}
\setlength{\headheight}{15pt}



% patch gauss macros for doing their work in `align'
% and other amsmath environments; see
% http://tex.stackexchange.com/questions/146532/
\usepackage{etoolbox}
\makeatletter
\patchcmd\g@matrix
 {\vbox\bgroup}
 {\vbox\bgroup\normalbaselines}% restore the standard baselineskip
 {}{}
\makeatother
\newcommand{\BAR}{%
  \hspace{-\arraycolsep}%
  \strut\vrule % the `\vrule` is as high and deep as a strut
  \hspace{-\arraycolsep}%
}

\pagestyle{fancy}

%subsections get indexed with letters
%\renewcommand{\thesubsection}{\thesection.\alph{subsection}}

%Navn og dato
\lhead{Macroeconomics III}
\chead{University of Copenhagen}
\rhead{\today}

\cfoot{Page \thepage\ of \totalpages}


%begynd dokumentet
\begin{document}
\begin{spacing}{1.213}

\title{ MACROECONOMICS \textbf{III} \\ \Large Winter 2016/2017 exam notes \\ \large \textit{University of Copenhagen - Department of Economics}  \\ 
\normalsize Kristian Urup Olesen Larsen \\
\normalsize Last updated \today  } % Supply information
%\author{} % for the title page.
\date{} % Use current date.
\maketitle % Print title page.
%\pagenumbering{roman} % Roman page number for toc
\setcounter{page}{1} % Make it start with "ii"

%\chapter{A Main Heading} % Make a "chapter" heading
\pagenumbering{arabic} % Start text with arabic 1

%--------------------------------------------------------------
%-------SKRIV HERUNDER----------------------------------------
%--------------------------------------------------------------
\pagebreak

\tableofcontents
\newpage

\section{Optimal control theory}
Optimal control theory is essentially a two step procedure, similar to the lagrange method. It's carried out as outlined below
\begin{itemize}
\item Set up the hamiltonian $H$
\item Applying the maximum principle $\left( \frac{\partial H}{\partial \textrm{state}} = -\dot{\lambda}(t), \ \frac{\partial H}{\partial \textrm{control}} = 0, \ \frac{\partial H}{\partial \lambda} = \dot{\textrm{state}} \right) $
\end{itemize}
These apply both in finite and infinite horizon settings and with a number of variations (below). OC problems consist of the following building blocks
\begin{itemize}
\item A \textbf{state} variable - the variables whose evolution is given by the LOM, most often $\dot{k}$, the state is always the dotted variable. Below this is called $y_t$
\item A \textbf{control} variable: the variable we make choices about, and the one we maximize w.r.t, in most cases $c(t)$. Below $u(t)$ is the control variable. 
\item A \textbf{Law of Motion} (LOM) An equation describing the evolution of the state variable over time. Below given by $f(t,y,u)$
\item An \textbf{objective function}, which is the function which is to be optimized over. Below $F(t,y,u)$
\item A set of \textbf{boundary conditions}. In the general case below, these are $y_0$ given and $y_T$ free.
\end{itemize}
Formally we can write this as 
\begin{align*}
&\underset{u(t)}{\textrm{max }} \int_0^{T} \overbrace{F(t,y,\underbrace{u}_{\mathclap{\textrm{control}}})}^{\textrm{objective}} \textrm{d}t \\
& \textrm{s.t. } \dot{y} = \overbrace{f(t,\underbrace{y}_{\mathclap{\textrm{state}}},u)}^{\textrm{LOM}} \\ 
&y_0 \textrm{ given}, y_t \textrm{ free}, u(t) \in U, \forall t \in [0;T]
\end{align*}
To solve this we write up the hamiltonian 
\begin{equation}
H(t,y,u,\lambda) = F(t,y,u) + \lambda(t) f(t,y,u) \quad (= \textrm{objective} + \lambda(t) \cdot \textrm{LOM})
\end{equation}
where $\lambda(t)$ is called the costate variable, and is interpreted as the shadow value of the state variable. Thus by the maximum principle we seek
\begin{align*}
&\frac{\partial H}{\partial u} = 0 && \frac{\partial H}{\partial y} = - \dot{\lambda} \\
& \frac{\partial H}{\partial \lambda} = \dot{y} && \underbrace{\lambda(T) = 0}_{\textrm{finite case}} \textrm{ or } \underbrace{\lim_{t\rightarrow \infty} y(t)\lambda(t) = 0}_{\textrm{infinite case}}
\end{align*}
It's important to note that the transversality condition (TVC) $\lambda(T)= 0 $ isn't general, but changes with the problems boundary conditions.

\paragraph{In the infinite case} we simply integrate as in $\int_0^{\infty} F(\cdot) \textrm{ d}t$, and change the TVC to the limit, for example $\lim_{T \rightarrow \infty} y(T) \geq 0$ or more commonly the one stated above.

\paragraph{Phase diagrams} are constructed by plotting the curves for $\dot{y}= 0 $ and $ \dot{u} = 0 $ in $(y,u)$ space. This is elaborated below.

\section{The Ramsey model}
The model is essentially a microfounded version of the Solow model. It was proposed to study how much income nations should save $\Rightarrow$ has savings as an endogenous variable (unlike Solow).

%\paragraph{Setup of the Ramsey model} consists of two groups: Households (HH) and firms. HH receive income from labor and renting out capital). They spend money on consumption and savings - these markets must clear. To begin with, only physical capital and no growth.

%\paragraph{HH} are infinitely lived, with a budget constraint of 
%\begin{equation}
%\dot{k} = r(t)k(t) + w_t - s_t
%\end{equation}
%they take $w_t$ and $r_t$ as given. The representative HH cares only about consumption, and discounts with rate $\rho$, so the lifetime utility (in the infinite case) is
%\begin{equation}
%U = \int_0^{\infty} e^{-\rho t} \cdot u(c_t) \textrm{ d}t
%\end{equation}
%which will be the optimization target for HH, they therefore solve
%\begin{align*}
%&\underset{c_t}{\textrm{max }} \int_0^{\infty} e^{-\rho t} \cdot u(c_t) \textrm{d}t \\
%& \textrm{s.t. } \dot{k} = r(t)k(t) + w_t - s_t, \qquad k_0 \textrm{ given}
%\end{align*}


%\subsection{Lecture 4}
\subsection{The Ramsey model - setup for households}
The Ramsey model describe the following economy
\begin{itemize}
\item Two types of agents: households and firms
\item Firms pay for labor and capital to produce output
\item HH get income from labor and renting out capital
\item HH spend money on consumption and savings
\item HH are infinitely lived, and solve a optimal control problem
\end{itemize}

The population is denoted $L(t)$ and grows at rate $n$, the number of households ($H$) is constant, and thus they have size $\frac{L(t)}{H}$ which grows with rate $n$ as well. HH's discount the future with $\rho$ and care only for consumption. We can then write a households lifetime utility as the discounted utility of consumption for an individual, times the number of individuals in a HH:
\begin{equation}
U(0) = \int_0^{\infty} e^{-\rho t} u(c_t) \frac{L(t)}{H} \textrm{ d}t
\end{equation}
where $c_t$ is consumption at time $t$ and $u(\cdot)$ is strictly increasing and concave, $C^2$ differentiable and satisfies INADA so that $\lim_{c \rightarrow \infty} u'(c) \rightarrow 0$ and $\lim_{c \rightarrow 0} u'(c) \rightarrow \infty$

\paragraph{Population growth} can be implemented by assuming $H=1$ (representative household) and $L(0)=1$ (normalization) so that $L(t) = e^{nt}$ whereby
\begin{align*}
U(0) &= \int_0^{\infty} e^{-\rho t} u(c_t) e^{nt} \textrm{ d}t \\
&= \int_0^{\infty} e^{-(\rho-n) t} u(c_t) \textrm{ d}t
\end{align*}
where it is standard to assume $\rho > n$ so that $U(0)$ is finite. The intuition behind this is that total (potential) consumption grows, even with constant $c$ because of population growth so if HH's are to impatient they'll be infinitely happy waiting forever before beginning to consume.

HH maximize utility subject to a budget constraint for each point in time, the standard budget constraint is 
\begin{equation}
c(t) + \dot{a}(t) + n a(t) = r(t)a(t) + w(t)
\end{equation}
Here $c$ and $w$ are standard - $a$ denotes assets. The $na(t)$ term originates from the growing population. Note that households are restricted by asset - not capital - holdings. This is because HH's could theoretically borrow from themselves through bonds, so that $A(t) = K(t) + B(t)$. Because HH's are identical, in equilibrium we'll have $B(t) = 0$ though. The budget constraint changes depending of the setup. Common modifications are taxes on $w_t$, $r_ta_t$ or financial intermediaries.

\paragraph{The no-ponzi-game (NPG) condition} is a restriction we impose to prevent infinite borrowing. Unlike the TVC it's not necessary for solving the problem, but we use it to enforce nice properties on the solution. It can be stated as 
\begin{equation}
\lim_{t \rightarrow \infty} a(t) e^{- \int_0^t (r(s)-n) \textrm{ d} s} \geq 0
\end{equation}

Thus it states that the present value of all further assets must be positive and not infinite (the "$\geq$" sign indicates that the value is in $\mathbb{R}^+$ with zero). 

\subsection{The Ramsey model - setup for firms }
For firms the procedure is standard profit maximization with constant returns to scale giving 
\begin{align*}
&R(t) = \frac{\partial F(K,L)}{\partial K} = \frac{\partial L \cdot F(\frac{K}{L},1)}{\partial K} = \frac{\partial F(\frac{K}{L},1)}{\partial \frac{K}{L}} = f'(k) \\
&w(t) =  \frac{\partial F(K,L)}{\partial L} = \frac{\partial L \cdot F(\frac{K}{L},1)}{\partial L} =\underbrace{F(\frac{K}{L},1) + L\frac{\partial F(\frac{K}{L},1)}{\partial \frac{K}{L}} \frac{\partial \frac{K}{L}}{\partial L}}_{\textrm{using the product rule}} =  f(k) - kf'(k)
\end{align*}
The only thing to take note of is that the profit function is sometimes modified with depreciation, and thus becomes
\begin{align*}
\pi_t = F(K_t,L_t) - w_tL_t - (r_t+\delta)K_t
\end{align*}
whereby the result for the real interest rate changes slightly, because of the $-\delta$.

\subsection{Solving the Ramsey model}
with the above we can state the HH problem as 
\begin{align*}
&\qquad\underset{c_t}{\textrm{max }} \int_0^{\infty} e^{-(\rho-n) t} u( c_t ) \textrm{ d}t 
\\
&\textrm{s.t.  } c(t) + \dot{a}(t) + n a(t) = r(t)a(t) + w(t) 
\\
&\textrm{\&  }  \lim_{t \rightarrow \infty} a(t) e^{- \int_0^t (r(s)-n) \textrm{ d} s} \geq 0 
\\
& \qquad a_0 \textrm{ given}
\end{align*}
This can be solved with optimal control by following the below standard procedure in 6 steps:
\begin{itemize}
\item[1)] Set up the hamiltonian $H = e^{-(\rho -n)t}u(c_t) - \lambda(t)(r_t a_t + w_t - n a_t - c_t)$
\item[2)] Find derivatives w.r.t the state $a_t$, the costate $\lambda$ and the control $c_t$ 
\begin{align}
&\frac{\partial H}{\partial c_t} = e^{-(\rho -n)t}u'(c_t) - \lambda(t) &&(=0) \label{eq: H1}
\\ 
&\frac{\partial H}{\partial a_t} = \lambda(t)[r_t -n] && (= - \dot{\lambda}(t) ) \label{eq: H2}
\\
& \frac{\partial H}{\partial a_t} = r_t a_t + w_t - n a_t - c_t && (=\dot{a}_t) \label{eq: H3}
\end{align}
\item[3)] differentiate (\ref{eq: H1}) w.r.t $t$ to get $-\dot{\lambda}_t = -(\rho-n)e^{-(\rho -n)t}u'(c_t) +e^{-(\rho -n)t}u''(c_t) \dot{c}_t$
\item[4)] Insert step 3 in (\ref{eq: H2}) so that 
\begin{align*}
&\lambda(t)[r_t -n] = -(\rho-n)e^{-(\rho -n)t}u'(c_t) +e^{-(\rho -n)t}u''(c_t) \dot{c}_t 
\end{align*}
\item[5)] Insert (\ref{eq: H1}) in step 4
\begin{align*}
&e^{-(\rho -n)t}u'(c_t)[r_t -n] = -(\rho-n)e^{-(\rho -n)t}u'(c_t) +e^{-(\rho -n)t}u''(c_t) \dot{c}_t 
\end{align*}
\item[6)] Reaarange the expression to get the Euler equation
\begin{align*}
& \frac{\dot{c(t)}}{c(t)} = \overbrace{\frac{-u'(c_t)}{c_t u''(c_t)}}^{\textrm{IES}}(r(t) - \rho)
\end{align*}
\end{itemize}
by assuming CRRA utility $u = \frac{c(t)^{1-\sigma)}}{1-\sigma}$ the Euler equation becomes even simpler, because of the constant IES 
\begin{equation}
\frac{\dot{c(t)}}{c(t)} = \frac{1}{\sigma}(r(t) - \rho)
\end{equation}

\subsection{Analysis in the Ramsey model}
\paragraph{Equilibrium conditions} are then that the HH to HH lending is 0 $B(t) = 0$ (id households are identical and one wants to borrow, they all will and thus borrowing becomes impossible), that the goods market clear $Y(t) = C(t) + \dot{K}(t)$ and finally that the price firms pay for capital $R(t)$ equals the interest received by lenders $r(t)$ so that the general equilibrium Euler equation is
\begin{equation}
\frac{\dot{c(t)}}{c(t)} = \frac{1}{\sigma}(f'(k(t)) - \rho)
\end{equation}

\begin{wrapfigure}[]{r}{0.3\linewidth}
\refstepcounter{figure}\label{fig: fprime}
\vspace{-15pt}
\def\svgwidth{\linewidth}
\input{fig1.pdf_tex}
 \footnotesize{\textbf{Figure \ref{fig: fprime}} The function $f'(k_t)$, notice that for a decrease in $k_t: k^* \rightarrow k'$ the value of $f'(k)$ increases}
\end{wrapfigure}

In addition in equilibrium it holds that that population can be normalized to one $L_t = L^{AG}(t)=1$ and thus $k_t = K_t = K^{AG}=a_t$ unless the model features financial intermediaries.

\paragraph{Phase diagrams} can be drawn from the Euler equation and the budget constraint. This process is easy if divided into analysis of each of the equations separately.

\begin{wrapfigure}[]{r}{0.415\linewidth}
\refstepcounter{figure}\label{fig: klocus}
\vspace{-15pt}
\def\svgwidth{\linewidth}
\input{fig2.pdf_tex}
{ \footnotesize{\textbf{Figure \ref{fig: klocus}} derived $c_t(k_t)$ decomposed to show that the $\dot{c}=0$ locus will be a "bulge".}}
\label{fig: kloc}
\end{wrapfigure}

Begin by seeing in the Euler equation that $\dot{c}_t = 0$ if and only if $f'(k_t)=\rho$, and this doesn't depend on $c_t$ so in a $(k,c)$ diagram the locus will be a vertical line.
Next notice that when $k_t<k^*$, $f'(k_t)>f'(k^*)$ so from the Euler equation it follows that $\dot{c}_t>0$. Therefore \textit{we conclude} that for $k_t<k^*$ (that is below the vertical line) $\dot{c}_t$ is positive and $c_t$ grows. Oppositely if $k_t>k^*$ it shrinks. (See figure \ref{fig: fprime})). The intuitive explanation for this is that when capital is low, the interest rate (MK$_K$) will be high relative to the impatience parameter $\rho$, making it attractive to invest. This shifts consumption from the future to the present, increasing consumption at time $t$. \marginpar{\footnotesize Is the intuition for $\dot{c}$ correct?}

\textbf{For the $\bf\dot{k}_t$ locus} we look at the budget constraint, where we set $\dot{k}_t=0$, so that it becomes $c(t) = r(t)a(t) + w(t) - n a(t)$. Now insert the results profit maximization results for $r_t$ and $w_t$, along with the equilibrium assumption $k_t=a_t$ and see that
\begin{align*}
c_t = f'(k_t)k_t+f(k_t)-k_t f'(k_t) - nk_t = f(k_t) - nk_t
\end{align*}
If we assume depreciation of capital, the $n$ is replaced by $(n+\delta)$ but analysis is the same. The above expression links $c_t$ and $k_t$, so we can draw it in a $(k,c)$ diagram. Because of the diminishing marginal returns of $f(k_t)$ this will constitute a bulge with two roots in $c_t$ (see figure \ref{fig: kloc} for intuition). To understand disequilibriums, imagine if $c_t> f(k_t)-n k_t$ keeping $k_t$ constant. Then the residual of $c_t$ must be captured by $\dot{a}_t = \dot{k}_t $, and this implies $\dot{k}_t < 0$ - capital is decreasing above the "bulge" and vice versa. Thus we can plot the final phase diagram with arrows indicating out-of-equilibrium movements.
\marginpar{\footnotesize Add intuitive explanation for $\dot{k}$}

\begin{figure}[h]
\centering
\refstepcounter{figure}\label{fig: phases}
\begin{minipage}{0.5\textwidth} % choose width suitably
\def\svgwidth{\columnwidth}
\input{fig3.pdf_tex}
{\footnotesize \textbf{Figure \ref{fig: phases}}
The final phase-diagram. Note that the saddle path can be constructed from the direction arrow.
\par}
\end{minipage}
\end{figure}

\paragraph{Changes in parameters} \marginpar{\footnotesize Add graphs with examples} are analysed simply by looking at the Euler equation, and the budget constraint with $\dot{k}_t = 0$ imposed. In case a tax or similar is changed, it often shift one or both curves.
\begin{itemize}
\item For permanent changes in period $0$ capital is fixed, so household jump vertically to the new saddle path at $t_0$, and then converge along the saddle path to the new equilibrium.
\item For temporary changes, households only move some of the distance to the new saddle path, before begining to converge back to the old saddle path. Again remember that $k_0$ is fixed to begin with (they start with a vertical jump) and be careful that in $t_0$ the arrows for motions outside of equilibrium are changed. 
\end{itemize}

\section{The OLG model}
\subsection{OLG - setup for Households}
The OLG model is a discrete time model, where agent live for two periods. The population is thus the sum of $L_t$ and $L_{t-1}$ where $L_t = (1+n)L_{t-1}$ describes the number of births in period $t$. They derive utility from consumption so that the lifetime utility of an individual born in period $t$ is 
\begin{equation}
U_t = u(c_{1t}) + \frac{1}{1+ \rho} u(c_{2t+1})
\end{equation}
when young agent supply 1 unit of labor at wage $w_t$, which is split between consumption $c_{1t}$ and savings $s_t$. When old agents cannot work, and thus only consume their savings, with interest $r_{t+1}$. That is in period 1 the budget is $w_t = c_{1t} + s_t$ and in period two $c_{2t+1} = (1+r_{t+1})s_t$. The budgets for each period can be combined into the intertemporal budget constraint by substituting out $s_t$, giving
\begin{align}
c_{1t} + \frac{1}{1+ r_{t+1}} c_{2t+1} = w_t
\end{align}
Along with the utility function, this constitute the maximization problem that is OLG
\begin{align*}
& \underset{c_{1t},c_{2t+1}}{\textrm{max }} u(c_{1t}) + \frac{1}{1+ \rho} u(c_{2t+1}) \\
& \textrm{s.t. } c_{1t} + \frac{1}{1+ r_{t+1}} c_{2t+1} = w_t
\end{align*}

\subsection{OLG - setup for firms}
like in the Ramsey model firms seek to maximize profit. It is standard to assume a Cobb-Douglas production function, whereby $r_t,w_t$ becomes (for derivation see the Ramsey model)
\begin{align*}
&r_t = f'(k) \\
&w_t = f(k) - kf'(k)
\end{align*}
It's worth noting that in order to derive a simple expression for $k_{t+1}(k_t)$ one typically arrives at an expression similar to $\frac{f(k_{t+1})}{1+f(k_{t+1})- \delta}$ - which requires $\delta =1$ to reduce further. 
\subsection{Solving the OLG model}
The model can be solved with the Lagrange method, giving the OLG equivalent of the Euler-equation from Ramsey, begin by setting up the lagrange equation
\begin{align*}
\mathcal{L}(c_{1t},c_{2t+1},\lambda) =  u(c_{1t}) + \frac{1}{1+ \rho} u(c_{2t+1}) - \lambda(c_{1t} + \frac{1}{1+ r_{t+1}} c_{2t+1} - w_t)
\end{align*}
Then taking first derivatives $\frac{\partial \mathcal{L}}{\partial c{1t}}, \ \frac{\partial \mathcal{L}}{\partial c{2t+1}}, \ \frac{\partial \mathcal{L}}{\partial \lambda}$ (all $=0$) and isolating $\lambda$ in the two first, makes it possible to derive the OLG version of the Euler equation
\begin{equation}
u'(c_{1t}) = \frac{1 + r_{t+1}}{1+ \rho}u'( c_{2t+1})
\end{equation}
By substituting in the expression for single period budgets we get an implicit expression of $s_t(w_t, r_{t+1})$
\begin{equation}
u'(w_t - s_t) = \frac{1 + r_{t+1}}{1+ \rho}u'( (1+r_{t+1})s_t)
\end{equation}
By assuming a "nice" utility function (i.e. $\log(\cdot))$ one can derive an actual expression for $s_t=s_t(w_t, r_{t+1})$.

\paragraph{A LOM for capital} can be derived after a short digression about aggregates in OLG. The aggregate capital accumulation in the economy is 
\begin{align*}
&K_{t+1} = \underbrace{S_t -K_t}_{\mathclap{\textrm{total savings of the young minus the dissaving of the old}}} + (1-\delta)K_t && \Leftrightarrow \\
& k_{t+1}(1+n) = s_t - \delta k_t
\end{align*}
From the firm maximization we can insert the profit maximization results for $r_{t+1}$ and $w_t$ (remember that it's $r_{t+1}$ not $r_t$!). We can use this, along with the expression for $s_t(w_t, r_{t+1})$ to derive a law of motion for capital
\begin{align*}
k_{t+1} &= \frac{1}{1+n} (s_t(w_t, r_{t+1}) - \delta k_t ) \\
& = \frac{1}{1+n} s_t((f(k_t)-f'(k_t)k_t, \underbrace{f'(k_{t+1})}_{\textrm{notice }k_{t+1}} ) - \delta k_t )
\end{align*}

\subsection{Analysis of the OLG model}
An important note to make is the concept of \textit{dynamic efficiency} (it's sometimes asked in exams). An economy is dynamically efficient if $r_t >n$. When this is not the case, the government can redistribute wealth between generations with a return of $1+n$ thus beating the market for savings. These situations arise in OLG because living generations cannot enter into contracts with the unborn.
\paragraph{Golden rule savings} (The maximum consumption on the BGP) can be determined from the $\dot{k}=0$ locus. Simply see that $c^* = f'(k^*)-nk^*$, which implied $f(k^G) = n$. For the actual level, start by plugging in $k_{t+1} = k_t = k^*$ in the capital accumulation function $k_{t+1} (k_t)$ which gives a level of $k^*$ as a function of parameters. With log-utility and CD production this becomes 
\begin{align*}
k^* = \left(\frac{1-\alpha}{(1+n)(2+ \rho)}\right)^{\frac{1}{\alpha-1}}
\end{align*}
and since $f'(k_t) = \alpha k_t ^{\alpha-1}$ we can derive that households choose savings so that 
\begin{align*}
f'(k^*) = \frac{\alpha}{1-\alpha}(1+n)(2+\rho)
\end{align*}
which is not generally bigger/smaller than the golden rule level $n$. This shows that in OLG decentralized solutions are not necessarily pareto optimal. Because of this OLG models can often be modified with a tax, without the loss of welfare taxes give in the Ramsey model.\marginpar{\footnotesize Add figure of $k_{t+1}(k_t)$ process}


\section{The RBC model}
The RBC model consists of a closed economy with no public sector and four perfect- competition markets (although often one or more are excluded).
\begin{align*}
& \bullet \textrm{ The goods market}
&& \bullet \textrm{ The labor market}
\\
& \bullet \textrm{ The capital (rental) market}
&& \bullet \textrm{ The bonds market}
\end{align*}
\textit{The supply side} each period chooses capital and labor input based on $w_t$ and $R_t$, they produce a good $y_t$ which can both be consumed and invested.
\\ \\
\textit{The demand side} have rational expectations and make best-possible predictions.

\subsection{RBC - setup for firms}
The output is produced with the production function which obey INADA conditions. \marginpar{\footnotesize Add comment on technology growth}
\begin{align*}
Y_t = A_t F(K_t, L_t)
\end{align*}
There's assumed to be an upward trend in factor productivity $X_t = \gamma X_{t-1}$ - this ensures long run growth. Capital evolves as is standard $K_{t-1} = I_t + (1-\delta) K_t$. Further because output is for either consumption or investment 
\begin{align*}
& Y_t = C_t + I_t
\end{align*}

\subsection{RBC - setup for households}
HH's split their time between leisure $L_t$ and labor $N_t$: $L_t + N_t = H$ (this implies endogenous labor supply in the model), normally normalized to 
\begin{align*}
L_t + N_t = 1
\end{align*}
Each individual lives forever and thus maximized \textit{expected} utility from a utility function of the from 
\begin{align*}
U_0 = E_0 \sum_{t=0}^{\infty} b^t U(C_t, L_t), \qquad 0<b<1
\end{align*}
Most often the problem is scaled to per-capita variables, and the conditions collapsed into a single equation.
\begin{align*}
&\underset{c_t, N_t, k_{t+1}}{\textrm{max}} E_0 \sum_{t=0}^{\infty} \beta^t u(c_t, L_t) \\
& \textrm{s.t. } c_t + \gamma k_{t+1} = w_t N_t + (1+R_t - \delta)k_t + \Pi_t
\end{align*}


\subsection{Solving the RBC model}
The model is solved with (a modified version of) the Lagrange method. 
\begin{itemize}
\item[1)] Begin by writing the Lagrange function (note the parentheses)

\begin{equation*}
\mathcal{L} = E_0 \left\{ \sum_{t=0}^{\infty} \beta^t \left[ u(c_t, L_t) + \lambda_t (w_t N_t + (1+R_t - \delta)k_t + \Pi_t - c_t - \gamma k_{t+1}) \right] \right\}
\end{equation*}

\item[2)] Now expand the lagrange function over two periods, for example $T$ and $T+1$ 
\small
\begin{align*}
\mathcal{L} &= E_T \left\{  \beta^{T} \left[ u(c_{T}, L_{T}) + \lambda_{T} (w_{T} N_{T} + (1+R_{T} - \delta)k_{T} + \Pi_{T} - c_{T} - \gamma k_{T+1}) \right] \right\} \\
& \quad + E_T \big\{  \beta^{T+1} [ u(c_{T+1}, L_{T+1}) + \lambda_{T+1} (w_{T+1} N_{T+1}  \\
& \qquad + (1+R_{T+1} - \delta)k_{T+1} + \Pi_{T+1} - c_{T+1} - \gamma k_{T+2}) ] \big\} 
\end{align*}
\normalsize
\item[3)] Now this expression shows that for the choice of $k_{t+1}$ there are effects from both period $T$ and $T+1$, but no further periods. Thus we can find first order conditions (note we can drop $E_{T}$ on period $T$ variables - also the derivative w.r.t $\lambda_T$ is left out because it just produces the budget constraint)
\begin{align*}
& (0=) \quad \frac{\partial \mathcal{L}}{\partial k_{T+1}} = -\beta^T \lambda_T \gamma + E_T \beta^{T+1}(1+ R_{T+1} -\delta)\lambda_{T+1}
\\
& (0=) \quad \frac{\partial \mathcal{L}}{\partial c_t} = \beta^T u'_c(c_t,\underbrace{1-N_t}_{\mathclap{\textrm{remember } L_t = 1- N_t}}) - \lambda_T
\\
& (0=) \quad \frac{\partial \mathcal{L}}{\partial N_{T+1}} = \beta^T u'_{N}(c_t, 1-N_t) + \lambda_T w_T
\end{align*}

\item[4)] Rearrange the FOC's to yield wages, labor supply etc. (in the slides Emiliano use that $\lambda_t$ is a shadow price so that $\lambda_t = u'_c(c_t,L_t)$). From FOC 1 this gives him the below)
\begin{align*}
&\beta^T u_c'(c_T,L_T) \gamma = \beta^{T+1}(1+E_T[R_{T+1}] - \delta)\beta u_c'(c_{T+1},L_{T+1}) && \Leftrightarrow \\
& \gamma u_c'(c_T,L_t) = \beta u_c'(c_{T+1},L_{T+1}) (1 + E_T[R_{T+1}] - \delta)
\end{align*}

\item[5)] See that the results for period $T$ generalizes to any $t$ without loss of generality.
\end{itemize}


\section{The nominal rigidity model(s)}
\subsection{Deriving a Philips curve (the Lucas way)}
Imagine a situation where producers observe prices $P_i$ in the economy, along with some average price level $P$, but they cannot separate price changes from changing demand, and price changes from changes in $P$.Then how will firms react to changes in level price level? We begin by deriving a Philips curve, assume that each household is also a producer, which maximizes 
\begin{equation}
U_i = C_i - \frac{1}{\gamma}L_i^{\gamma}
\end{equation}
Where $C_i$ is consumption, $Y_i$ is demand for good $i$ and $L_i$ is labor supply. Also assume linear production technology $Y_i = L_i$ and $P = \bar{P}_i$. Lastly note that total (real) revenues of a good is $\frac{P_i}{P}Y_i$ which must be equal to consumer $i$'s total consumption $C_i$, thus 
\begin{align*}
&\underset{Y_i}{\textrm{max }} U_i && \Rightarrow
&&& \frac{\partial }{\partial Y_i} \frac{P_i}{P}Y_i - \frac{1}{\gamma}Y_i^{\gamma}  = 0 &&&& \Rightarrow \\ 
&\frac{P_i}{P} = Y_i^{\gamma-1}
\end{align*}
Taking log's of this (note $p = \log(P)$ etc) and rearranging gives us  
\begin{align*}
y_i = \frac{1}{\gamma -1 }(p_i - p) = \frac{1}{\gamma -1 }r_i
\stepcounter{equation}\tag{\theequation}\label{eq: ybasic}\
\end{align*}
Further not that from the quantity theory of money $y = m-p$, so (\textcolor{red}{how?}) aggregate demand on logarithmic form is given by \marginpar{\footnotesize Figure out how/ why questions} 
\begin{align*}
y_i = y + z_i - \eta (p_i - p) = m-p + z_i - \eta (p_i - p)
\end{align*}
$z_i$ is a taste shock and $\eta>0$. Since each producer only observes $p_i$ and $p$ and not their decomposition $= p + (p_i - p)$, they have to guess the size of $r_i = p_i - p$ (the relative prices). We can derive $E[r_i | p_i ]$ by assuming $m \sim N(E(m),V_m)$ and $z \sim N(0,V_z)$ (\textcolor{red}{why?}), and using the rule \marginpar{\footnotesize Rasmus derives variance in a different (possibly more correct) way} 
\begin{align*}
\begin{pmatrix}
x \\
y
\end{pmatrix} \sim
N\begin{pmatrix}
\begin{pmatrix}
E(x) \\
E(y)
\end{pmatrix} ,
\begin{pmatrix}
\sigma^2_{1} & \sigma_{12}  \\
\sigma_{21}  & \sigma^2_{2}
\end{pmatrix}
\end{pmatrix}
\Rightarrow
E(x|y) = E(x) + \frac{\sigma_1^2}{\sigma_2^2}(y - E(y))
\end{align*}
by which it follows that 
\begin{align*}
E(r_i|p_i) = E(r_i) + \frac{V_r}{V_r + V_p}(p_i - E(p))
\end{align*}
where the nominator is simply the variance of $r_i$ and the denominator can be found from $p_i = p + (p_i -p) = p+ r_i$ which implies $var(p_i) = V_p + V_r$. Now recall the previous expression for $y_i$ in (\ref{eq: ybasic}) and assume \textit{certainty equivalence} to replace $r_i$ with $E(r_i|p_i)$ 
\begin{align*}
y_i = \frac{1}{\gamma -1}r_i = \frac{1}{\gamma -1}\frac{V_r}{V_r + V_p} (p_i - E(p))
\end{align*}
where it's further used that $E(r_i)=E(p_i)-p = 0$. Summing over all producers gives
\begin{equation} \label{eq: lucasSupply}
y = b(p-E(p))
\end{equation}
Which is the \textit{Lucas supply curve}. \textbf{Note} that this shows a positive relationship between output and prices.

\paragraph{Implementing general equilibrium} is simply a matter of setting (\ref{eq: lucasSupply}) (the Lucas supply curve) equal to demand $y = m - p$, which yields
\begin{align*}
&p = \frac{1}{1+b}m +\frac{b}{1+b} E(p) \\
& y = \frac{b}{1+b} m - \frac{b}{1+b} E(p)
\end{align*}
From the first of these results we can take expectations to show $E(p) = E(m)$. With this, and by decomposing $m$ into its observed and unobserved parts $m = E(m) + (m-E(m))$ we get
\begin{align*}
&p = \frac{1}{1+b}(E[m] + (m-E[m] )) + \frac{b}{1+b} E[m] = E[m] + \frac{b}{1+b}(m- E[m]) \\
& y = \frac{b}{1+b} ( E[m]- (m - E[m])) - \frac{b}{1+b} E[m] = \frac{b}{1+b} (m- E[m])
\end{align*}
From these equations we can analyse the effects of known and unknown shocks to $m$. If the shock is unobserved $E[m]$ stays constant, while $m$ changes. If the shock is observed $m - E[m]$ is constant, while $E[m]$ changes.

\paragraph{The fundamental representation} of $b$ ($b$ as a function of $V_z$ and $V_m$) will finish off the model. Combining the single consumer Lucas supply with added $p$'s: $y_i = b(p_i-p) + b(p - E(p))$ with the demand curve gives
\begin{align}
& b\underbrace{(p_i-p)}_{r_i} + \underbrace{b(p - E(p))}_{y} = y + z_i + \eta \underbrace{ (p_i - p)}_{r_i} \\
& z_i = (b+ \eta)r_i \\
& r_i = \frac{z_i}{b+ \eta} \rightarrow V_r = \frac{V_z}{(b+ \eta)^2}
\end{align}
Likewise from the $p(m, E[m])$ expression we have 
\begin{align*}
V_p = \frac{V_m}{(1+b)^2}
\end{align*}
and thus we can finally insert these in the expression for $b$ 
\begin{equation}
b = \frac{1}{\gamma -1 } \frac{ \frac{V_z}{(b+\eta)^2)} }{ \frac{V_z}{(b+\eta)^2} + \frac{V_m}{(1+b)^2} } = \frac{V_z}{V_z + \frac{(b + \eta)^2}{(1+b)^2} V_m}
\end{equation}

\paragraph{Deriving a Philips curve} requires assuming some LOM for $m$, if we take $m_t = m_{t-1} + c + u_t$ we have $E(m_t)=m_{t-1}+c$ and thus 
\begin{align*}
&p_t  = E[m_t] + \frac{b}{1+b}(m_t- E[m_t]) = m_{t-1} + c + \frac{b}{1+b} u_t \\
& y_t  = \frac{b}{1+b} u_t
\end{align*}
Now taking the first difference of $p_t$ will yield inflation 
\begin{align*}
\pi_t &= m_{t-1} - m_{t-2} + \frac{b}{1+b}(u_t -u_{t-1}) = \underbrace{c + u_{t}}_{\mathclap{m_{t-1}-m_{t-2}}} + \frac{b}{1+b}(u_t -u_{t-1} ) 
\\
&= c + \frac{1}{1+b}u_t + \frac{b}{1+b}u_{t-1}
\end{align*}
Now use that $y_t  = \frac{b}{1+b} u_t \Rightarrow \frac{1}{b}y_t = \frac{1}{1+b}u_t$ and $E_{t-1}[\pi_t] = c +\frac{b}{1+b}u_{t-1}$ to get 
\begin{equation}
\pi_t = \frac{1}{b}y + E_{t-1}[\pi_t]
\end{equation}
This shows that output and inflation is positively correlated again, which is a proper Philips curve. We see that only unobserved shocks affect real variables. Imagine a shift in the average money growth at period $t$: $c \rightarrow c_t$, if the change is unobserved $E_{t-1}[\pi_t] = c + \frac{b}{1+b}u_{t-1}$ and thus 
\begin{align*}
\pi_t = c_t - \underbrace{c + E_{t-1}[\pi_t]}_{= \frac{b}{1+b}u_{t-1}} + \frac{1}{b}y_t
\end{align*}
If the change is observed, however $E_{t-1}[\pi_t] = \textcolor{purple}{c_t} + \frac{b}{1+b}u_{t-1}$ and thus inflation is unchanged.

\subsubsection{Next step: modeling price setting behavior}
The Lucas model predicts that high demand volatility $V_m$ should reduce the real effects of changes in aggregate demand. This hypothesis however has less explanatory power compared to the nominal rigidities hypothesis. Again log-demand is given by $y_i = y - \eta(p_i - p)$ also the price set by a producer will be positively related to aggregate demand (because higher output drives up wages through the assumption of increasing disutility of labor)
\begin{align*}
&p_i - p = \phi(m - p), \qquad \phi \geq 0 
  \stepcounter{equation}\tag{\theequation}\label{eq: phiandy}\ 
\\
& p_i = \phi m + (1-\phi) p
\end{align*}
If prices are set every period, assume that $m$ is random, prices must be set conditional on the available information $I$. 
\begin{align*}
p_i = \phi E[m|I] + (1-\phi) E[p|I]
\end{align*}
Assuming that everyone behaves in the same way $p_i = p$ and thus $E[p|I]= E[m|I]$. Equilibrium condtions are then 
\begin{align*}
&p = p_i = E[p|I] = E[m|I] \\
& y = m-p = m- E[m|I]
\end{align*}
\subsection{The Fischer model}
In the Fischer model each firm/HH sets prices for more than one (typically two) periods, and only a fraction (typically $\frac{1}{2}$) of firms can change prices each period. The firms can set different prices for $t$ and $t+1$, but cannot revise their decisions before $t+2$.

\begin{figure}[h]
\caption{}
\centering
\begin{minipage}{0.8\textwidth} % choose width suitably
\def\svgwidth{\columnwidth}
\input{drawing.pdf_tex}
{\footnotesize 
\\
Outline of how the equilibrium values of $p_t$ and $y_t$ are derived in the Fischer model - the actual derivation can be found below.
\par}
\end{minipage}
\end{figure}
\marginpar{\footnotesize Redo figure?}
\FloatBarrier

Let $p_t^i$ be the price set at period $t$ conditional on the information available in period $t-i$. Thus 
\begin{align*}
&p_t = \frac{1}{2}(p_t^1 + p_t^2) 
\end{align*}
Using the fact that $p_i = \phi m + (1-\phi) p$ 
\begin{align*}
& p_t^* = \phi m_t + (1-\phi)\frac{1}{2}(p_t^1 + p_t^2)
\end{align*}
Price setters are rational and use all information, so $p_t^1 = E_{t-1}[p_t^*]$ and $p_t^2 = E_{t-2}[p_t^*]$ whereby we have
\begin{align}
&p_t^1 = \phi E_{t-1}[m_t] + (1-\phi) \frac{1}{2} \underbrace{(p_t^1 + p_t^2)}_{\textrm{known at time }t-1} \label{eq: p1org} \\
& p_t^2 = \phi E_{t-2}[m_t] + (1-\phi) \frac{1}{2} (E_{t-2}[p_t^1] +\underbrace{ p_t^2}_{\mathclap{\textrm{known at time }t-2}}) \label{eq: p2org}
\end{align}
\paragraph{Solving the model} begins with solving for $p_t^1$ in (\ref{eq: p1org}) above which will give 
\begin{align*}
&p_t^1 = \phi E_{t-1}[m_t] + (1-\phi) \frac{1}{2} (p_t^1 + p_t^2) \\
& p_t^1 \left[
1- (1-\phi)\frac{1}{2}
\right] = E_{t-1}[m_t] + (1-\phi) \frac{1}{2}  p_t^2 \\
& p_t^1 = \frac{2\phi}{1+\phi} E_{t-1}[m_t] + \frac{1-\phi}{1+\phi} p_t^2
  \stepcounter{equation}\tag{\theequation}\label{eq: p1mod}\ 
\end{align*}
Likewise for $p_t^2$ we get from (\ref{eq: p2org}) 
\begin{align*}
p_t^2 = \frac{2\phi}{1+\phi} E_{t-2}[m_t] + \frac{1-\phi}{1+\phi} E_{t-2}[p_t^1]   \stepcounter{equation}\tag{\theequation}\label{eq: p2mod}\ 
\end{align*}
Now take $E_{t-2}[p_t^1]$ in (\ref{eq: p1mod}) 
\begin{align*}
E_{t-2}[p_t^1] = E_{t-2}
\left[
\frac{2 \phi}{1+\phi} E_{t-1}[m_t] + \frac{1-\phi}{1+\phi} p_t^2
\right]
= \frac{2 \phi}{1+\phi} E_{t-2}[m_t] + \frac{1-\phi}{1+\phi} p_t^2
\end{align*}

Plug in this in original expression for $p_t^2$ in (\ref{eq: p2org})
\begin{align*}
&p_t^2 = \phi E_{t-2}[m_t] + (1-\phi) \frac{1}{2} (E_{t-2}[p_t^1] + p_t^2) \\
& = \phi E_{t-2}[m_t] + (1-\phi) \frac{1}{2} \left( \frac{2\phi}{1+\phi} E_{t-2}[m_t] + \frac{1-\phi}{1+\phi} p_t^2 + p_t^2\right) \\
& = \phi E_{t-2}[m_t] +  \frac{(1-\phi)\phi}{1+\phi} E_{t-2}[m_t] + (1-\phi) \frac{1}{2}\left( \frac{1-\phi}{1+\phi}+1 \right) p_t^2 \\
& = \phi E_{t-2}[m_t] +  \frac{(1-\phi)\phi}{1+\phi} E_{t-2}[m_t] +\left( \frac{1-\phi}{1+\phi} \right) p_t^2 \\ 
&p_t^2 \left(1- \frac{1-\phi}{1+\phi} \right) = E_{t-2}[m_t]\left( \frac{2\phi}{1+ \phi} \right) \\
&p_t^2 = E_{t-2}[m_t]
\end{align*}
Now plug this into (\ref{eq: p1mod})
\begin{align*}
 p_t^1 &= \frac{2\phi}{1+\phi} E_{t-1}[m_t] + \frac{1-\phi}{1+\phi} p_t^2 \\
&  = \frac{2\phi}{1+\phi} E_{t-1}[m_t] + \frac{1-\phi}{1+\phi} E_{t-2}[m_t] \\
& = \frac{2\phi}{1+\phi} E_{t-1}[m_t] + \overbrace{\left(1 - \frac{2\phi}{1+ \phi} \right)}^{\frac{1+\phi-2\phi}{1+\phi}}  E_{t-2}\\ 
& = E_{t-2}[m_t] + \frac{2\phi}{1+\phi} \left( E_{t-1}[m_t] - E_{t-2}[m_t] \right)
\end{align*}
To derive the equilibrium values recall the definition $p^*_t = \frac{1}{2}(p_t^1 + p_t^2)$ and that $y = m-p$, thus 
\begin{align*}
p_t^* &= \frac{1}{2} \left(\overbrace{E_{t-2}[m_t]}^{p_t^2} + \overbrace{ E_{t-2}[m_t] + \frac{2\phi}{1+\phi} \left( E_{t-1}[m_t] - E_{t-2}[m_t] \right)}^{p_t^1} \right) 
\\
& = E_{t-2}[m_t] + \frac{\phi}{1+\phi} (\textcolor{purple}{ E_{t-1}[m_t] - E_{t-2}[m_t]}) 
\\
y^* &= m_t - \left( \overbrace{E_{t-2}[m_t]  + \underbrace{\frac{\phi}{1+\phi}}_{=1-\frac{1}{1+\phi}} (E_{t-1}[m_t] - E_{t-2}[m_t])}^{p_t^*} \right)
\\
& = m_t  - \left( E_{t-2}[m_t] + (E_{t-1}[m_t] -E_{t-2}[m_t]) - \frac{1}{1+\phi}(E_{t-1}[m_t] - E_{t-2}[m_t])  
\right) 
\\
& = \textcolor{blue}{m_t - E_{t-1}[m_t]} + \frac{1}{1+ \phi}(\textcolor{purple}{E_{t-1}[m_t] -E_{t-2}[m_t]})
\end{align*}
Which we can interpret as dividing shocks in \textcolor{blue}{unanticipated} shocks that have real effect just like in the Lucas model and \textcolor{purple}{anticipated} shocks (they're anticipated in the sense they're "discovered between $t-2$ and $t-1$ and thus half the firms have time to react). The anticipated shocks only affected prices in Lucas, but now, due to nominal rigidities they also affect the real economy.

\textbf{Note} that a higher $\phi$ implies that anticipated shocks are increasingly affecting prices, and less so $y$ - this is because $\phi$ measures the degree of real rigidity - if $\phi$ is large, prices are very responsive to demand shocks. 

\paragraph{Stabilization in the Fischer model}
is based on the postulation that $y_t = m_t - p_t + v_t$. Here $m_t$ represents policy effects on aggregate demand, $v_t$ is non-policy shocks (so we simply split up the old $m_t$ in two, $m_t^{\textrm{old}} = m_t^{\textrm{new}} + v_t$). With this the equilibrium conditions naturally become
\begin{align*}
p_t^* &= E_{t-2}[m_t + v_t] + \frac{\phi}{1+\phi} ( E_{t-1}[m_t + v_t] - E_{t-2}[m_t + v_t]) 
\\
y^* &= m_t + v_t - E_{t-1}[m_t + v_t] + \frac{1}{1+ \phi}(E_{t-1}[m_t+ v_t] -E_{t-2}[m_t+ v_t])
\end{align*}
The non-policy shocks are assumed to follow a random walk $v_t = v_{t-1} + \epsilon_t$. When the government chooses $m_t$ they use all available information, and sets $m_t$ as 
\begin{align*}
m_t = a_1 \epsilon_{t-1} + a_2 \epsilon_{t-2} + ... + a_{n} \epsilon_{t-n} + ... 
\end{align*}
Note the rule for $m_t$ assumes a linear function of $\epsilon$'s - implicitly this gives a specific form of preferences for society. Now do the following: 1) rearrange the expression for $y^*_t$, 2) write up $m_t+v_t$, 3) derive $E_{t-1}[m_t + v_t]$ and 4) derive $E_{t-2}[m_t + v_t]$.

\begin{itemize}
\item[1)] We first rewrite $y_t^*$ by collecting the $E_{t-1}$'s that is 
\begin{align*}
y_t^* &= m_t + v_t - E_{t-1}[m_t + v_t] + \frac{1}{1+ \phi}E_{t-1}[m_t+ v_t] - \frac{1}{1+ \phi}E_{t-2}[m_t+ v_t] \\
&= m_t + v_t - \frac{\phi}{1+ \phi} E_{t-1}[m_t + v_t] - \frac{1}{1+ \phi} E_{t-2}[m_t + v_t]
\end{align*}

\item[2)] See that because $v_t = v_{t-1} + \epsilon_t$ we have
\begin{align*}
m_t + v_t = v_{t-1} + \epsilon_t + a_1 \epsilon_{t-1} + a_2 \epsilon_{t-2} + ... + a_{n} \epsilon_{t-n} + ...
\end{align*}

\item[3)] Take $E_{t-1}$ of step 3

\begin{align*}
E_{t-1}[m_t + v_t] &= E_{t-1}[\overbrace{v_{t-1} + \epsilon_t}^{v_t} + \overbrace { a_1 \epsilon_{t-1} + a_2 \epsilon_{t-2} + ... + a_{n} \epsilon_{t-n} + ...}^{m_t \textrm{ (all known at }t-1)}] \\
& = v_{t-1} + m_t \\
& = \underbrace{v_t - \epsilon_t}_{\mathclap{\textrm{recall }v_t = v_{t-1} + \epsilon_t }} + m_t
\end{align*}

\item[4)] Now for $E_{t-2}$ of step 3
\begin{align*}
E_{t-2}[m_t + v_t] &= E_{t-2}[v_{t-1} + \epsilon_t +  a_1 \epsilon_{t-1} + a_2 \epsilon_{t-2} + ... + a_{n} \epsilon_{t-n} + ...] \\
& = E_{t-2}[\overbrace{v_{t-2}+ \epsilon_{t-1}}^{v_{t-1}} + \epsilon_t +  a_1 \epsilon_{t-1} + a_2 \epsilon_{t-2} + ... + a_{n} \epsilon_{t-n} + ...] \\
& = v_{t-2} -a_1 \epsilon_{t-1} + \overbrace{a_1 \epsilon_{t-1} + a_2 \epsilon_{t-2} + ... + a_{n} \epsilon_{t-n} + ...}^{m_t} \\
& = \overbrace{v_{t-1} - \epsilon_{t-1}}^{v_{t-2}} -a_1 \epsilon_{t-1} + m_t \\
& = v_t - \epsilon_t - (1+a_1)\epsilon_{t-1} + m_t 
\end{align*}
\end{itemize}
With these results we can go back to step 1, and insert the derived expectations. Note that $1- \frac{\phi}{1+\phi}-\frac{1}{1+\phi} = 0$ thus we're left with
\begin{align*}
y_t^* &= m_t + v_t - \frac{\phi}{1+ \phi} E_{t-1}[m_t + v_t] - \frac{1}{1+ \phi} E_{t-2}[m_t + v_t] \\
& = m_t + v_t - \frac{\phi}{1 + \phi}(v_t - \epsilon_t + m_t) - \frac{1}{1+ \phi} (v_t - \epsilon_t - (1+a_1)\epsilon_{t-1} + m_t) \\
& = \epsilon_t + \frac{1+ a_1}{1+ \phi} \epsilon_{t-1}
\end{align*}

\subsection{The Calvo model}
The Calvo model makes price setting stochastic, so that firms set new prices with probability $\alpha$ each period. Let $x_t$ be the updated prices in period $t$ (these are not optimal prices because firms take into account the risk of having fixed prices in the future). Thus the overall price level is 

\begin{align} \label{eq: calvoaggprice}
p_t = \alpha x_t + (1-\alpha) p_{t-1}
\end{align}
Whenever a firm can set new prices, they set $x_t$ as a discounted average of future optimal $p^*_{t+i}$'s, weighted with the probability that the firms still haven't changed prices $i$ periods ahead, that is 
\begin{align} \label{eq: calvo1}
x_t = [1- \beta(1-\alpha)] \sum_{j=0}^{\infty} \beta^j(1-\alpha)^j E_t [p_{t+j}^*]
\end{align}
Write out the first period where $j=0$ ("scorporate") to get 
\begin{align*}
x_t = [1- \beta(1-\alpha)]p_t^* + \underbrace{[1- \beta(1-\alpha)] \sum_{j=1}^{\infty} \beta^j(1-\alpha)^j E_t [p_{t+1+j}^*]}_{E_t[x_{t+1}] \textrm{ by comparison with (\ref{eq: calvo1})}}
\end{align*}
Now subtract $p_t$ from both sides. On the LHS add and subtract $p_{t-1}$ and on the RHS add and subtract $\beta(1-\alpha)p_t$ to get 
\begin{align*}
(x_t-p_{t-1})-(p_{t}-p_{t-1}) &= [1- \beta(1-\alpha)](p_t^*- p_t) \\ 
& + [1- \beta(1-\alpha)] \sum_{j=1}^{\infty} \beta^j(1-\alpha)^j (E_t [p_{t+1+j}^*] - p_t)
\end{align*}
Subtract $p_{t-1}$ on both sides of (\ref{eq: calvoaggprice}) to get an expression for inflation $p_t - p_{t-1} = \pi_t = \alpha(x_t - p_{t-1}) \Rightarrow x_t - p_{t-1} = \frac{\pi_t}{\alpha}$. Furthermore (\ref{eq: phiandy}) states that $p_t^* - p_t = \phi y_t$, so we have 
\begin{align*}
&(\frac{\pi_t}{\alpha})-(\pi_t) = [1- \beta(1-\alpha)](\phi y_t) \\ 
& + [1- \beta(1-\alpha)] \sum_{j=1}^{\infty} \beta^j(1-\alpha)^j (E_t [p_{t+1+j}^*] - p_t) \\
&\pi_t = \frac{\alpha}{1-\alpha}[1- \beta(1-\alpha)]\phi y_t + \beta E_t[\pi_{t+1}]
\end{align*}
which is an expectational difference equation (EDE). In the slides for lecture 12 their solution is shown in more detail. Assuming that the proper conditions for a convergent solution are met, it can be solved by repeated substitution on $\pi_{t+1}$, yielding

\begin{equation}
\pi_t = \frac{\alpha \phi}{1- \alpha}[1- \beta(1-\alpha)] \sum_{j=0}^{\infty} \beta^j E_t [y_{t+j}]
\end{equation}
This is the \textit{new Keynesian Philips curve}. A number of variations of the price setting scheme exists. These can be found in the slides for lecture 14, but are left out since deriving the solutions follows the same procedure as above, with a different expression for $p_t^*$. \marginpar{\footnotesize  Include alternative setups of the Calvo model}


\subsection{The Blanchard- Kiyotaki model for monopolistic price setting}
The slides run through a graphical analysis of monopoly price setting with menu costs - the conclusion is essentially that the costs of changing prices might force monopoly producers to keep prices fixed and out of optimum, see slides for a more in depth analysis. 
\\ \\ 
Assume a static model where a finite number of firms $m$ produce $m$ differentiable goods. These goods are imperfect substitutes, and firm $i$ is price setter for good $i$. 

\paragraph{Households} solve a utility maximization problem of choosing consumption of each good. Note that $X_{(X_i)}$ is notation to indicate $X=f(X_i)$.
\begin{align*}
&\textrm{max } U = C_{(C_i)}^{\gamma} \left( \frac{M}{P_{(P_i)}} \right)^{1-\gamma} - \frac{1}{\beta}N^{\beta} \\
& \textrm{s.t. } \sum P_i C_i + M = M_0 + WN + \sum \Pi_i
\end{align*}
where otherwise notes $\sum = \sum_{i=1}^m$. Further the total consumption is set as a CES function of $C_i$'s (similarly for $P_i$'s). Thus 
\begin{align*}
&C_{(C_i)} = m^{\frac{1}{1-\theta}} \left( \sum C_i^{\frac{\theta-1}{\theta}} \right)^{\frac{\theta}{\theta -1}} \\ 
& P_{(P_i)} = \left( \frac{1}{m} \sum P_i^{1-\theta} \right)^{\frac{1}{1-\theta}}
\end{align*}

\paragraph{Solving this model for HH's} can be done with lagrange (and some finesse with derivatives of sums), begin with 
\begin{align*}
\mathcal{L} = C_{(C_i)}^{\gamma} \left( \frac{M}{P_{(P_i)}} \right)^{1-\gamma} - \frac{1}{\beta}N^{\beta} + \lambda \left( M_0 + WN + \sum \Pi_i - \sum P_i C_i - M \right)
\end{align*}
The derivative of $\mathcal{L}$ w.r.t $C_i$ is thus 
\begin{align*}
\frac{\partial \mathcal{L}}{\partial C_i} &= \gamma C_{(C_i)}^{\gamma -1 }C'_{(C_i)} \left( \frac{M}{P_{(P_i)}}\right)^{1 - \gamma} - \lambda \underbrace{P_i}_{\mathclap{\substack{\textrm{remember } \sum P_i C_i = P_1C_1 + P_2C_2 +  ... ) \\  \textrm{ so the derivative w.r.t some } i \textrm{ is just } P_i}}} \qquad (=0)
\\
&= \gamma \left( \frac{M}{P_{(P_i)}C_{(C_i)}} \right)^{1-\gamma} C'_{(C_i)} - \lambda P_i
\end{align*}
Now see that $C'(C_i)$ is 
\begin{align*}
C'_{(C_i)} &= m^{\frac{1}{1-\theta}} \frac{\theta}{\theta -1} \left( \sum C_i^{\frac{\theta -1 }{\theta}} \right)^{\frac{\theta}{\theta-1}-1} \frac{\theta-1}{\theta} C_i^{\frac{\theta -1}{\theta} -1} 
\\
& = m^{\frac{1}{1-\theta}} \left( \sum C_i^{\frac{\theta -1 }{\theta}} \right)^{\frac{1}{\theta-1}} C_i^{\frac{-1}{\theta}} 
\\
&= \underbrace{m^\frac{-1}{\theta}}_{\mathclap{ \substack{
\textrm{b.c. } C^{\frac{1}{\theta}} \textrm{ gives } m^{\frac{1}{(1-\theta)\theta}} \\
\textrm{and } m^{\frac{1}{(1-\theta)\theta}}m^{\frac{-1}{\theta}} = m^{\frac{1}{1-\theta}}
}}} 
C_{(C_i)}^{\frac{1}{\theta}} C_i^{\frac{-1}{\theta}}
\\
& = \left( \frac{C_{(C_i)}}{m C_i} \right)^{\frac{1}{\theta}}
\end{align*}
Thus we're left with the first FOC
\begin{align*}
\gamma \left( \frac{M}{P_{(P_i)}C_{(C_i)}} \right)^{1-\gamma} \left( \frac{C_{(C_i)}}{m C_i} \right)^{\frac{1}{\theta}} = \lambda P_i
\end{align*}
The second FOC is w.r.t $M$
\begin{align*}
\frac{\partial \mathcal{L}}{\partial M} &= C_{(C_i)}^{\gamma}(1-\gamma)M^{-\gamma}P_{(P_i)}^{\gamma}P_{(P_i)}^{-1} - \lambda  \qquad (= 0) && \Leftrightarrow \\
& (1-\gamma)\left( \frac{M}{P_{(P_i)}C_{(C_i)}} \right)^{-\gamma} = \lambda P_{(P_i)}
\end{align*}
and lastly the third FOC is w.r.t $N$
\begin{align*}
&-N^{\beta-1} + \lambda W \qquad (=0) && \Leftrightarrow &&& N^{\beta-1}= \lambda W
\end{align*}

\paragraph{The demand for $\bf C_i$}can be found from the two first FOC's (isolate $\lambda$ and equate the two) now dropping the subscripts indicating that indies depend on individual values ($X_{(X_i)}=X$).
\begin{align*}
&P_i^{-1}\gamma \left( \frac{M}{PC}\right)^{1- \gamma} \left(\frac{C}{m C_i}\right)^{\frac{1}{\theta}} = (1-\gamma)\left( \frac{M}{PC}\right)^{-\gamma} P^{-1} \\
& P_i = \frac{\gamma}{1-\gamma} \frac{M}{C} \left(\frac{C}{m C_i}\right)^{\frac{1}{\theta}}
\end{align*}
To aggregate, recall that $\left( \frac{C}{mC_i}\right)^{\frac{1}{\theta}}$ is $C'(C_i)$. If we aggregate the change-contributions of all $C_i$'s of course the total change of consumption is $1$, in the sense that $C$ changes $1:1$ with changes in $C_i$'s. Thus 
\begin{align*}
&P = \frac{\gamma}{1-\gamma}\frac{M}{C} && \Rightarrow &&& M= \frac{\gamma-1}{\gamma} PC 
\end{align*}
By plugging $M$ back into the expression for $P_i$ we thus have 
\begin{align*}
P_i = P \left( \frac{C}{mC_i} \right)^{\frac{1}{\theta}} \\
C_i = \left( \frac{P_i}{P} \right)^{-\theta} \frac{C}{m}
\end{align*}
which gives the demand for good $i$. Further we can derive labor supply etc. which except for a lot of algebra simply involves imposing the equilibrium condition $Y=C$. 

\paragraph{Firms solve} a somewhat non-standard profit maximization problem where their maximization problem is given by 
\begin{align*}
&\underset{P_i, N_i, Y_i}{\text{max }} \Pi_i = P_i Y_i - W N_i \\
&\textrm{s.t. } Y_i = C_i = \left( \frac{P_i}{P} \right)^{-\theta} \frac{C}{m} \\
& \ \   \quad Y_i = N_i^{\alpha}, \qquad 0 < \alpha < 1 
\end{align*}
Inserting the constraints reduces the problem to one-dimensional optimization, which is simply a matter of solving
\begin{align*}
\underset{P_i}{\text{max }} \Pi_i = P_i\left( \frac{P_i}{P} \right)^{-\theta} \frac{C}{m} - W \left[ \left( \frac{P_i}{P} \right)^{-\theta} \frac{C}{m} \right]^{\frac{1}{\alpha}}
\end{align*}

\section{The optimal policy rule model}
These models are concerned with endogenizing policy decisions chosen by a government or a central bank. They consider the policy maker to be a rational agent constrained by private sector actions and institutions. 

\paragraph{Perrson and Tabellini}'s model is given by two equations, a reduced form demand curve and a new-keynesian philips curve, respectively
\begin{align*}
&\pi = m + v + \mu \\
& x = \theta + \pi - \pi^e - \epsilon
\end{align*}
where $v, \epsilon$ and $\mu$ are white noise processes. The timing is as follows 
\begin{itemize}
\item A monetary policy rule is set
\item All agents observe $\theta$ 
\item $\pi^e$ is formed by the private sector
\item agents observe $v$ and $\epsilon$
\item $\mu, \pi$ and $x$ are realized
\end{itemize}
First notice that due to the timing, we have from the demand equation $E[\pi| \theta] = E[m|\theta]$. Inserting inflation in the NKPC thus gives
\begin{align*}
x &= \theta + m + v -\mu - E[m|\theta] - \epsilon \\
& = \theta + \underbrace{m - E[m]}_{\mathclap{\text{only unanticipated shocks have real effects}}} + v + \mu - \epsilon
\end{align*}
We then introduce a social loss function, describing societys preferences for stabilization in $x$ and/or $\pi$. Let $\bar{x}, \bar{\pi}$ be target values, and assume the loss function takes a quadratic form 
\begin{align*}
L(x, \pi) = \frac{1}{2}[(\pi - \bar{\pi})^2 + \lambda (x - \bar{x})^2 ]
\end{align*}
where $\lambda$ measures societys relative preferences for output shocks relative to shocks in inflation. It can be shown that if societys preferences are quadratic, the optimal policy for $m$ is linear, so lets assume 
\begin{align*}
m = \psi + \psi_{\theta} \theta + \psi_{\epsilon} \epsilon + \psi_v v
\end{align*}
\subsection{Solution under credibility}
Under the assumption of credibility $\pi^e = E(\pi| \theta) = E(m|\theta)$ we therefore have $\pi^e = \psi + \psi_{\theta} \theta$. Inserting this in the original equations gives
\begin{align*}
\pi &= m + v + \mu \\
& = \psi + \psi_{\theta} \theta + \psi_{\epsilon} \epsilon + \psi_v v + v + \mu \\
&= \psi + \psi_{\theta} \theta + \psi_{\epsilon} \epsilon +  (1 + \psi_v)v + \mu \\
x & = \theta + \pi - \pi^e - \epsilon \\
&= \theta + (\psi_{\epsilon} -1) \epsilon + (\psi_v +1 )v + \mu
\end{align*}
Because the monetary policy must be announced before any other events, we cannot optimize the actual loss function under credibility. Instead we maximize the expected loss. First plug in $x$ and $\pi$
\begin{align*}
E[L(x, \pi)] &= \frac{1}{2}E[(\pi - \bar{\pi})^2 + \lambda (x - \bar{x})^2 ] \\
& = \frac{1}{2}E[(\psi + \psi_{\theta} \theta + \psi_{\epsilon} \epsilon + (1 + \psi_v)v + \mu - \bar{\pi})^2  \\
&  +\lambda (\theta + (\psi_{\epsilon} -1) \epsilon + (\psi_v +1 )v + \mu - \bar{x})^2 ]
\end{align*}

Under the assumption that all stochastic variables are orthogonal to one another taking squares simplify a lot, since their products then become 0 in expectation. This leaves us with
\begin{align*}
E[L(x, \pi)] &= \frac{1}{2} [ \psi^2 + \psi_{\theta}^2 \sigma^2_{\theta} + \psi_{\epsilon}^2 \sigma^2_{\epsilon} + (1 + \psi_v)^2 \sigma^2_v + \sigma^2_{\mu} + \bar{\pi}^2 - 2\psi \bar{\pi}
\\
& \quad + \lambda(\sigma^2_{\theta} + (\psi_{\epsilon} -1)^2 \sigma^2_{\epsilon} + (\psi_v +1 )^2\sigma^2_v + \sigma^2_{\mu} + \bar{x}^2)  ]
\end{align*}
Now to find optimal parameters for the policy rule, simply find first order conditions 
\begin{align*}
&\frac{\partial E[L]}{\partial \psi} = \frac{1}{2} (2\psi - 2 \bar{\pi}) &&\rightarrow \psi = \bar{\pi} \\
& \frac{\partial E[L]}{\partial \psi_{\theta}} = \psi_{\theta}\sigma^2_{\theta} && \rightarrow \psi_{\theta} = 0 \\
& \frac{\partial E[L]}{\partial \psi_{\epsilon}} = \psi_{\epsilon}\sigma^2_{\epsilon} + \lambda(\psi_{\epsilon}-1)\sigma^2_{\epsilon} && \rightarrow \psi_{\epsilon} = \frac{\lambda}{1+ \lambda} \\
& \frac{\partial E[L]}{\partial \psi_{v}} = (1+\psi_v )\sigma^2_v + \lambda(1+ \psi_v)\sigma^2_v && \rightarrow \psi_v = -1
\end{align*}
With these we can state the optimal policy rule by replacing the generic parameters with the optimal ones, that is 
\begin{align*}
m^{\text{opt}} = \bar{\pi} - v + \frac{\lambda}{1+ \lambda} \epsilon
\end{align*}
Inserting the optimal $m$ in the expressions for $x$ and $\pi$ gives the actual values of inflation and output under an optimal policy with credibility.
\begin{align*}
&\pi^C = \bar{\pi} + \frac{\lambda}{1+\lambda} \epsilon + \mu \\
& x^C = \theta - \frac{1}{1+ \lambda}\epsilon + \mu
\end{align*}

\subsection{Solution without credibility}
When the government conducts discretionary monetary policy, it can observe private sector expectations and $\theta$ before announcing $m$ - this effectively allows the government to select inflation directly. This implies a new timing 
\begin{itemize}
\item Private agents observe $\theta$
\item $\pi^e$ is formed in the private sector as $\pi^e = E[\pi|\theta]$
\item $\epsilon$ is observed 
\item The government sets $m$, and thereby realizes $\pi$ and $x$
\end{itemize}
Because the private sector forms expectations before knowing what $m$ will be, they will internalize the governments ex post (after timeline) incentives. Now the private sector and the government is playing a game, and the solution will be a Nash equilibrium where
\begin{align}
&\frac{\partial L}{\partial \pi} = 0, \quad \textrm{given } \pi^e, \epsilon \\
& \textrm{Expectations are rational } \pi^e = E(\pi|\theta)
\end{align}
Because the government can wait, it doesn't have to take expectations of the loss function, but can instead maximize it directly. Further because the government can effectively set $\pi$, the NKPC is modified (no shocks now)
\begin{align*}
x&= \theta + \pi - \pi^e - \epsilon \\
\pi &= m
\end{align*}
To solve the model, begin by deriving the socially optimal $\pi$, plug $x$ into the loss function and find the FOC w.r.t $\pi$ 
\begin{align*}
\frac{\partial L}{\partial \pi} &= \frac{\partial }{\partial \pi} \left(  \frac{1}{2}[(\pi - \bar{\pi})^2 + \lambda(\overbrace{\theta + \pi - \pi^e - \epsilon}^{x} - \bar{x})^2 ] \right) \\
& = \pi - \bar{\pi} + \lambda(\theta + \pi - \pi^e - \epsilon - \bar{x}) && (=0) \\
\end{align*}
Which reduces to 
\begin{equation} \label{eq: optuncredpi}
\pi = \frac{1}{1+ \lambda} \bar{\pi} + \frac{\lambda}{1+ \lambda}(- \theta + \pi^e + \epsilon + \bar{x})
\end{equation}
Now find the private sectors expected value of $\pi$ by taking conditional expectations of (\ref{eq: optuncredpi})
\begin{align*}
&\pi^e = E(\pi|\theta) = \frac{1}{1+ \lambda} \bar{\pi} + \frac{\lambda}{1+ \lambda}(- \theta + \pi^e + \bar{x}) && \Leftrightarrow \\ 
&\pi^e \left( \frac{1}{1+ \lambda} \right)= \frac{1}{1+ \lambda} \bar{\pi} + \frac{\lambda}{1+ \lambda}(- \theta + \bar{x}) && \Leftrightarrow \\
& \pi^e =\bar{\pi} + \lambda(\bar{x} - \theta) 
\end{align*}
Now we can derive the equilibrium inflation and output under discretionary policy. To get inflation insert $\pi^e$ in the expression for the optimal inflation
\begin{align*}
\pi^D &= \frac{1}{1+ \lambda} \bar{\pi} + \frac{\lambda}{1+ \lambda}(- \theta + \pi^e + \epsilon + \bar{x}) \bigg|_{\pi^e =\bar{\pi} + \lambda(\bar{x} - \theta) } \\ 
& = \frac{1}{1+ \lambda} \bar{\pi} + \frac{\lambda}{1+ \lambda}(- \theta + \bar{\pi} + \lambda(\bar{x} - \theta) + \epsilon + \bar{x}) \\
& = \bar{\pi} + \underbrace{\lambda(\bar{x}- \theta)}_{\textrm{inflation bias}} + \frac{\lambda}{1+ \lambda} \epsilon
\end{align*}
And likewise if we insert $\pi^D$ and $\pi^e$ in our original expression for $x$ we get $x^D$
\begin{align*}
x^D &= \theta + \pi - \pi^e - \epsilon \big|_{\pi^D, \pi^e} \\
&= \theta +  \overbrace{\bar{\pi} + \lambda(\bar{x}- \theta) + \frac{\lambda}{1+ \lambda} \epsilon}^{\pi^D} - \overbrace{(\bar{\pi}+ \lambda(\bar{x} - \theta))}^{\pi^e} - \epsilon \\
& = \theta - \frac{1}{1+\lambda} \epsilon
\end{align*}
Notice that the discretionary policy inflicts an inflation bias, but disregarding $\mu$ (which cannot be controlled by policy) $x^C = x^D$. Thus the discretionary policy increases inflation without benefiting the attempts of output stabilization. We can interpret the results as a government trying to boost output with "surprise inflation" which clearly isn't possible.  


\subsection{Characterizing the optimal central banker}
Instead of conducting it's own discretionary policy, the government has the choice of hiring an independent central banker. The banker has to be hired ex ante, and thus losses must be evaluated in expected values. The central banker conducts his/her own discretionary policy. Further the choice of banker depends on his/her parameter $\lambda^B \in [0;\lambda]$ and the "value" of the banker should be evaluated against the governments preferences $\lambda$. This implies the government wants to solve 
\begin{align*}
\underset{\lambda^B}{\textrm{min }} E[L(\pi^D, x^D, \lambda^B)] &= \frac{1}{2}E \bigg[ \left(
\bar{\pi} + \lambda^B(\bar{x}- \theta) + \frac{\lambda^B}{1+ \lambda^B} \epsilon - \bar{\pi} \right)^2 \\
& \qquad + \lambda \left( 
 \theta - \frac{1}{1+\lambda^B} \epsilon - \bar{x}
\right)^2
\bigg] \\
& = \frac{1}{2}\bigg[ (\lambda^B)^2(\bar{x}^2 + \sigma^2_{\theta}) + \left(\frac{\lambda^B}{1+ \lambda^B}\right)^2 \sigma^2_{\epsilon}  \\
& \qquad + \lambda \left(\sigma^2_{\theta} + \frac{1}{(1+\lambda^B )^2}\sigma^2_{\epsilon} + \bar{x}^2 \right) \bigg] 
\end{align*}
that is- minimize the social loss evaluated with the governments preferences $\lambda$, under discretionary rule by the independent central bank. The first derivative w.r.t to $\lambda^B$ will be 
\begin{align*}
\frac{\partial E[L(\pi^D, x^D, \lambda^B)]}{\partial \lambda^B} = \lambda^B(\bar{x}^2 + \sigma^2_{\theta}) + \frac{\lambda^B}{(1+\lambda^B)^3} \sigma^2_{\epsilon} - \lambda \frac{1}{(1+ \lambda^B)^3} \sigma^2_{\epsilon} \qquad (= 0)
\end{align*}
which has no analytic solution. Instead of deriving the optimal $\lambda^B$ we'll characterize the solution by evaluating the derivative in different values, namely $0$ and $\lambda$, see 
\begin{align*}
&\frac{\partial E[L(\pi^D, x^D, \lambda^B)]}{\partial \lambda^B} \bigg|_{\lambda^B = 0} = \lambda \sigma^2_{\epsilon} && >0 \\
& \frac{\partial E[L(\pi^D, x^D, \lambda^B)]}{\partial \lambda^B} \bigg|_{\lambda^B = \lambda} = \lambda(\bar{x}^2 + \sigma^2_{\theta}) && <0 
\end{align*}
It can be shown that $E[L(\pi^D, x^D, \lambda^B)]$ has only one minimum between $0$ and $\lambda$, this this shows that the optimal central banker (optimal $\lambda^B$) lies between $\lambda^B= 0$ and to governments own preferences $\lambda^B = \lambda$. 

\begin{figure}[h]
\centering
\refstepcounter{figure}\label{fig: CB}
\begin{minipage}{0.8\textwidth} % choose width suitably
\def\svgwidth{\columnwidth}
\input{EL.pdf_tex}
{\footnotesize \textbf{Figure \ref{fig: CB}}
Characterization of the optimal central banker - note how the sign of the derivatives imply an optimal central banker who has $0 < \lambda^B < \lambda$.
\par}
\end{minipage}
\end{figure}

\subsection{An optimal contract for the banker}

\subsection{Pegged currencies}

\subsection{Comparing losses}

\end{spacing}
\end{document}


