In order to study collusion in auctions we return to the private value auctions. In particular we study a second price auction with $N$ bidders the set of which we denote $\mathcal{N}$. Bidders valuations are distributed on $[0,\omega_i]$ and follow a distribution $F_i$ which can vary between bidders.

We consider a case where the subset $\mathcal{I} \in \mathcal{N}$ of all bidders have joined a bidding ring. That is bidders $1,2,...,I$ are in the ring while $I+1,...,N$ are not.
\\ \\
Define also a variable $Y_1^{\mathcal{S}}$ to be the highest valuation in the subset of bidders $\mathcal{S} \in \mathcal{N}$. $Y_1^{\mathcal{I}}$ is thus the highest signal among the bidding ring members. 
\\ \\ 
Bidders in the ring needs to coordinate to identify one representative for their ring, who will bid to win the item, while all others submit shill bids. In this way the ring can earn profit from exploiting the difference in perception about the number of bidders in the auction. 

CONTINUE FROM SLIDE 14