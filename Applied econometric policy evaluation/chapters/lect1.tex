The main focus of the course is the estimation of \textit{©ausal} effects of policy interventions. In other words the central question to answer is how people react to exogenous events. In this context the ideal setup is a randomized experiment where the effect of a treatment $D$ is measures w.r.t some outcome $y$. To achieve this a population should be randomly split in two, with one half getting the treatment $D=1$ and the other not getting it $D=0$. The size of interest is then the effect size $\delta = E[y_1]- E[y_0]$. In an OLS equation we will estimate $\delta$ as
\begin{equation}
y_i = \alpha + \beta D_i + u_i
\end{equation}
where the critical assumption is that $E[D|u]=0$. Throughout the course there are three core questions which should also be asked and answered,
\begin{itemize}
\item[1.] What is the causal relation of interest?
\item[2.] What is the ideal experiment that would capture the causal effect?
\item[3.] What is the identification strategy?
\end{itemize}
As well as a fourth question which is good to consider
\begin{itemize}
\item[4.] What is the mode of statistical inference?
\end{itemize}
The course lectures usually covers some theory, and shows an example of how this can be applied in real research.
